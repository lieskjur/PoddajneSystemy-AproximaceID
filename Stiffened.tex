	\begin{frame}
		\frametitle{Poddajné systémy - Pohybové Rovnice}

		\begin{equation*}
			\bm{M}\bm{\dot{v}} + \bm{c} = \bm{\tau}
			\;,\quad 
			\bm{\tau} = \bm{p} + \bm{B}\bm{u}
		\end{equation*}
		Rozdělíme zrychlení na řízená zrychlení $\bm{\dot{v}}_c$ a poddajná zrychlení $\bm{\dot{v}}_s$. Stejně rozdělíme sloupce matice hmotnosti, což nám dovolí úpravu
		\begin{equation*}
			\bm{M}\bm{\dot{v}} = \bm{M}_{:c} \bm{\dot{v}}_c + \bm{M}_{:s} \bm{\dot{v}}_s
		\end{equation*}
		Vektor pasivních silových účinků v kinematických dvojicích můžeme rozdělit na akumulační a disipační členy
		\begin{equation*}
			\bm{p} = \bm{A}\bm{q} + \bm{D}\bm{v}
		\end{equation*}
		S výslednou pohybovou rovnicí
		\begin{equation*}
			\bm{M}_{:c} \bm{\dot{v}}_c + \bm{M}_{:s} \bm{\dot{v}}_s + \bm{c} = \bm{A}\bm{q} + \bm{D}\bm{v} + \bm{B}\bm{u}
		\end{equation*}
	\end{frame}

	\begin{frame}
		\frametitle{Poddajné systémy - Poddajnost}
		Vnímáme-li akumulační ůčinky v $i$-té kin. vazbě jako zobecněné síly způsobené deformací elementu MKP, kde uzlové posuvy jsou nahrazeny zobecněnými parametry $\bm{q}$
		\begin{equation*}
			\bm{A}_i\bm{q}_i
			=
			\int_{V} \bm{N}_i^T  \, \bm{\sigma}_i \, dV
			\;,\quad 
			\bm{\sigma}_i = \bm{E}_i \bm{\varepsilon}_i
			\;,\quad 
			\bm{\varepsilon}_i = \bm{N}_i \, \bm{q}_i 
		\end{equation*}
	\end{frame}

	\begin{frame}
		\frametitle{Poddajné systémy - Poddajnost}
		Když uvažujeme těleso jako tuhé, říkáme, že jeho odezva na změnu zatížení $\Delta\bm{\tau}$ je taková změna napětí $\Delta\bm{\sigma}$, která zamezí podstatné deformaci tělesa $\Delta\bm{\varepsilon} \approx \bm{0} \,,\; \bm{\dot{\varepsilon}} \approx \bm{0} \,,\; \bm{\ddot{\varepsilon}} \approx \bm{0}$
		\\\vspace{1em}
		Budeme-li chtít zanedbat okamžitý vliv poddajnosti v pohybové rovnici (na úrovni zrychlení) dosadíme $\bm{\dot{v}}_s = \bm{0}$, jehož splnění bude zajištěno fiktivní změnou ``poddajných'' parametrů $\Delta\bm{q}_s$ ($\bm{\sigma} \propto \bm{q}_s$).
		\begin{equation*}
			\bm{M}_{:c} \bm{\dot{v}}_c + \bm{c} = \bm{p} + \bm{A}_{:s} \Delta\bm{q}_s + \bm{B}\bm{u}
		\end{equation*}
	\end{frame}

	\begin{frame}
		\frametitle{Poddajné systémy - Optimalizační problém}

		\begin{itemize}
			\item \textbf{Optimalizované parametry}\\
			mezi optimalizované parametry zařadíme i volná zrychlení
			\begin{equation*}
				\bm{x} = \begin{bmatrix} \bm{u} \\ \Delta\bm{q}_s \end{bmatrix}
			\end{equation*}
			\item \textbf{Cílová funkce}\\
			\begin{equation*}
				\bm{P}
				=
				\begin{bmatrix}
					\bm{1} & \bm{0} \\
					\bm{0} & \bm{0}
				\end{bmatrix}
				\;,\quad
				\bm{q} = \bm{0}
			\end{equation*}
		\end{itemize}
	\end{frame}

	\begin{frame}
		\frametitle{Poddajné systémy - Optimalizační problém}
		
		\begin{itemize}
			\item \textbf{Nerovnice přípustných řešení}\\
			\begin{equation*}
				\bm{a}
				=
				\begin{bmatrix}
					\bm{\tau}_{eq} \\
					\bm{u}_{min} \\
					% \Delta\bm{q}_{s_{min}}
					\bm{q}_{s_{min}} - \bm{q}_s
				\end{bmatrix}
				\;,\quad 
				\bm{A}
				=
				\begin{bmatrix}
					\bm{B} & \bm{A}_{:s} \\
					\bm{1} & \bm{0} \\
					\bm{0} & \bm{1}
				\end{bmatrix}
				\;,\quad
				\bm{b}
				=
				\begin{bmatrix}
					\bm{\tau}_{eq} \\
					\bm{u}_{max} \\
					% \Delta\bm{q}_{s_{max}}
					\bm{q}_{s_{max}} - \bm{q}_s
				\end{bmatrix}
			\end{equation*}
			kde
			\begin{equation*}
				\bm{\tau}_{eq} = \bm{M}_{:c} \bm{\dot{v}}_c + \bm{c} - \bm{p}
			\end{equation*}
			a
			\begin{equation*}
				\bm{E}_i\bm{N}_i : \bm{q}_i \mapsto \bm{\sigma}_i
			\end{equation*}
			pomocí kterého můžeme stanovit $\bm{q}_{s_{min}}$ a $\bm{q}_{s_{max}}$
		\end{itemize}
	\end{frame}