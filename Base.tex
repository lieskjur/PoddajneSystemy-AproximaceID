\begin{frame}
	\frametitle{Kvadratické programování}
	Úlohou je nalézt minimum kvatratické cílové funkce pro parametry $\bm{x}$ z množiny přípustných řešení popsané lineárními nerovnicemi.
	\begin{align*}
		\text{minimize:}& \quad \frac{1}{2}\bm{x}^T\!\bm{P}\bm{x} + \bm{q}^T\!\bm{x} \\ 
		\text{subject to:}& \quad \bm{a} \leq \bm{A}\bm{x} \leq \bm{b}
	\end{align*} \cite{osqp}
	Kvadratické programování patří do kategorie konvexního programování, kde lokální minim cílové funkce je také minimem globálním. \cite{convex}
\end{frame}

\begin{frame}
	\frametitle{Pohybové rovnice}
	
	\begin{equation*}
		\bm{M} \bm{\dot{v}} + \bm{c} = \bm{\tau}
		\;,\quad \bm{\tau} = \bm{p} + \bm{B}\bm{u}
	\end{equation*}
	\begin{itemize}
		\item [$\bm{M}$] - matice hmotnosti
		\item [$\bm{\dot{v}}$] - vektor nezávislých zrychlení
		\item [$\bm{c}$] - vektor účinků dostředivého a corriolisova zrychlení
		\item [$\bm{\tau}$] - silové účinky
		\item [$\bm{p}$] - pasivní silové účinky
		\item [$\bm{B}$] - manipulační matice
		\item [$\bm{u}$] - vektor vstupů
	\end{itemize}
\end{frame}